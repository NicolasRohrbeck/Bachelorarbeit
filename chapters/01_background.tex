\chapter{Theoretical Background \label{ch:background}}
This chapter aims to give an introduction into the \acf{BEXUS} programme as a whole and the \acf{SETH} experiment as a part of this programme. Next the need for the \acf{ADS} is motivated and the measurement methods of the accelerometer and magnetometer explained.

\section{\acs{SETH} and the \acs{BEXUS} programme \label{sec:seth_and_bx_programme}}

The first balloon of the \ac{BEXUS} programme, \textit{BEXUS 1} launched on November 25$^{\mathrm{th}}$ 2002 at 15:53 UTC. Since then every year except 2003, 2020 and 2022 saw balloons being launched from the designated launch site \acf{ESRANGE} near the Swedish city of Kiruna \cite{IAC-08.E.1.1.4}\cite{bexus-campaign-history}.\\
The \acf{BX} programme is a bilateral programme between the \acf{DLR} and \acf{SNSA}. It offers university student teams the opportunity to build their own experiment and have it fly on a stratospheric balloon. The programme spans around one year and is constructed as a scaled-down version of a real space mission. The \acf{SETH} takes part in the 16th \ac{BEXUS} cycle and is expected to fly between the 3$^\mathrm{rd}$ of October 2025 and 13$^\mathrm{th}$ of October 2025 form \ac{ESRANGE}.\\

The purpose of \ac{SETH} is to investigate the angular dependency of particle radiation in the upper atmosphere. To this end a means of measuring the experiments attitude while on the gondola is required. 

\section{Accelerometers \label{sec:accelerometers}}

\section{Magnetometers \label{sec:magnetometers}}
To determine the gondola's heading, a magnetometer is used just like a compass. There space based magnetometers \cite{space-based-magnetometers-review}.