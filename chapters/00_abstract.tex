\begin{otherlanguage}{ngerman}
\addchap{Kurzzusammenfassung}
In dieser Arbeit wird die Implementation des \acf{ADS} für das \ac{BEXUS}-Experiment \ac{SETH} beschrieben. \acs{REXUS}/\acs{BEXUS} ist ein bilaterales Programm des \acf{DLRde} und \acf{SNSA}, welches es Teams von Studierenden ermöglicht mit einem selbstgebauten Experiment innerhalb ungefähr eines Jahres so gut wie alle Stationen einer echten Raumfahrtmission mitzuerleben. Das \ac{SETH} Experiment nimmt am 16. Zyklus des \acs{REXUS}/\acs{BEXUS} Programmes teil und wird vorraussichtlich zwischen dem 03. Oktober 2025 und 13. Oktober 2025 aus der \ac{ESRANGE} bei der schwedischen Stadt Kiruna auf einem Stratosphärenballon starten.\\
Da es das Hauptziel von \ac{SETH} ist, die Winkelabhängigkeit des Regener-Pfotzer-Maximums zu bestimmen, braucht es einen zuverlässigen Weg um Roll- und Nickwinkel sowie die Blickrichtung der Gondel zu Bestimmen. Um diese Anforderungen zu erfüllen, wurde das \ac{ADS} als platzeffizienter Sensor aus einem Accelerometer und Magnetometer entworfen. Das Accelerometer wird genutzt, um Roll- und Nickwinkel zu bestimmen, das Magnetometer in Analogie zu einem Kompass, um die Blickrichtung zu bestimmen. Es wird gezeigt, dass die drei orthogonalen Sensorachsen eines jeden Sensors unter einem linearen Skalenfakror und Nullpnktverschiebung leiden. Um diese zu bestimmen werden drei Kalibrationsmethoden verglichen und die koeffizienten für Skalenfaktor und Nullpunktverschebung numerisch bestimmt. Die beste Kalibrationsmethode wird ermittelt, indem die relative Abweichung der kalibrierten Datensätze vom Optimalwert untersucht wird. Die optimalen Koeffizienten werden genutzt, um Roll- und Nickwinkel, sowie die Blickrichtung eines Wetterballonfluges auszuwerten.
\end{otherlanguage}

\addchap{Abstract}
In this thesis the implementation of the \acf{ADS} for the \acf{BEXUS} experiment \acf{SETH} is presented. The \acs{REXUS}/\acs{BEXUS} programme is a bilateral programme between the \acf{DLR} and \acf{SNSA}. It offers teams of university students the opportunity to build their own experiment and have it fly on a stratospheric balloon. \ac{SETH} will fly between the 3$^\mathrm{rd}$ of October 2025 and 13$^\mathrm{th}$ of October 2025 from the \acf{ESRANGE} near the Swedish city of Kiruna.\\ Because it is the main objective of \ac{SETH} to measure the angular dependency of the Regener-Pfotzer-Maximum of radiation intensity in the atmosphere, a means to know roll, pitch and heading at all times is needed. The \ac{ADS} is a small size \acs{PCB} that has been custom made to fulfil this need. The combination of an accelerometer and magnetometer allow for accurate measurement of the three needed angles. It is shown, that the sensors need to be calibrated for a scale factor and linear offset on each of the three orthogonal axes. The best fit parameters for these coefficients are chosen from a selection of three calibration methods by determining the relative difference from the optimal value. The optimal coefficients are used to calibrate the data gathered on a weather balloon flight and results for pitch, roll and heading shown and explained.