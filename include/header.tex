\documentclass[a4paper,openright,parskip,listof=totoc]{scrreprt}
%%%%%%%%%%%%%%%%%%%%%%%%%%%%%%%%%%%%%%%%%%%%%%%%%%%%                                                          %
% Festlegen der Dokumentenklasse und allgemeine Konfiguration                 %
%                                                                             %
% ------------------                                                          %
% mögliche Optionen:                                                          %
% ------------------                                                          %
% - twoside: zweiseitiges Layout (linke und rechte Seiten)                    %
% - openright: Kapitel beginnen auf der rechten Seite                         % 
%	- 10pt | 11pt | 12pt: Standard-Schriftgröße (11pt ist Voreinstellung)     %
% - DIV??: Abstand des Textes zum Rand                                        %
%	- a4paper: Papierformat DIN A4                                            %
% - landscape: Querformat                                                     %
%	- parskip: vertikaler Absatzabstand anstatt Einzug wie im amerikanischen  %
% - leqno: Formelnummerierung links statt rechts                              %
% - fleqn: Formeln links statt zentriert                                      %
% - draft, final: Entwurf oder finale Version                                 %
% - smallheadings | normalheadings | bigheadings: Größe der Überschriften     %
% - chapterprefix: die sonst ausgeblendeten Prefixe "Kapitel X", "Anhang Y",  %
%                  etc. werden dargestellt                                    %
% - liststotoc: Abbildungs- und Tabellenverzeichnis erscheinen auch im        %
%               Inhaltsverzeichnis                                            %
% - bibtotoc: Literaturverzeichnis erscheint im Inhaltsverzeichnis            %
%                                                                             %
% ----------------                                                            %
% mögliche Pakete:                                                            %
% ----------------                                                            %
% - scrartcl: kleine Dokumente, keine Kapitel                                 %
% - scrreprt: große Dokumente, extra Titelseite, Zusammenfassung und          %
%             Inhaltsverzeichnis, Kapitel                                     %
% - scrbook: sehr große Dokumente, zweiseitiger Satz                          %
% - scrlttr2: Briefe                                                          %
%%%%%%%%%%%%%%%%%%%%%%%%%%%%%%%%%%%%%%%%%%%%%%%%%%%%%%%%%%%%%%%%%%%%%%%%%%%%%%%

% Sonderzeichen wie deutsche Umlaute vom Schriftsatz ansinew % (unter Linux latin1 nehmen) 
\usepackage[utf8]{inputenc} 
\usepackage{verbatim}
\usepackage{listings}
\usepackage{pdflscape}
\usepackage{pdfpages}
\usepackage[T1]{fontenc} % Sorgt dafür, dass Wörter mit Umlauten getrennt werden.
\usepackage{ae,aecompl}  % Type 1 Zeichensatz, sieht besser aus als PDF
\usepackage{fontawesome} %fuer \faLink
\usepackage[main=english, ngerman]{babel} % neudeutsche Silbentrennung, Übersetzungen 
\usepackage{eurosym}
\usepackage[printonlyused]{acronym}  % Liste der Akronyme
\usepackage{siunitx} % fuer SI-Einheiten

% Mathepakete
\usepackage[intlimits]{amsmath}
\usepackage{amssymb,amsthm,amstext} 
\usepackage{amsfonts}

% Tabelle mit Zeilenumbrüchen
\usepackage{makecell}

% Pakete fr die bessere Nutzung von Tabellen
\usepackage{
  array,
  booktabs,
  dcolumn, % auf Dezimaltrennzeichen ausgerichtete Spalten in tabular
  tabularx,
}

% kann eine Barriere fr Float-Umgebungen erzeugen mit \FloatBarrier
\usepackage{placeins}

\usepackage[version=4,arrows=pgf]{mhchem}
% Float-Klasse
\usepackage{float}

% spezielle Optionen zur Veräderung der itemize- und enumerate-Umgebungen
\usepackage{paralist}

% Paket zum Einbinden von Grafiken
%\usepackage[pdflatex]{graphicx}			% bei Ausgabe als PDF
\usepackage{graphicx}						% bei Ausgabe als DVI

% Einbinden von mehreren Bildern in eine figure-Umgebung
\usepackage{subfigure}

% einige Sonderzeichen wie Grad Celsius, Copyright, etc.
\usepackage{textcomp} 

% zur schöneren Darstellung von physikalischen Einheiten
%\usepackage[thinspace,spaceqspace]{fancyunits} 

% Um Bilder im fließtext zu platzieren (bei word wäre es "passend")
\usepackage{wrapfig}

% Tabelle mit Seitenumbruch
\usepackage{longtable}

% für Farben im allgemeinen
\usepackage{color}

% für die Hintergrundfarbe einzelner Zellen in Tabellen
\usepackage{colortbl}

% Landscape mode
\usepackage{lscape}

% Paragraphen kommentieren
\usepackage{comment}


% --- Farbdefinitionen ----------------------------------------
%\definecolor{wordblue}{rgb}{0.55,0.7,0.89}
\definecolor{wordblue}{rgb}{0.9,0.9,0.9}
\definecolor{lightgrey}{rgb}{0.9,0.9,0.9}
\definecolor{verylow}{rgb}{0.0,1.0,0.0}
\definecolor{low}{rgb}{1.0,1.0,0.0}
\definecolor{medium}{rgb}{1.0,0.6,0.0}
\definecolor{high}{rgb}{1.0,0.0,0.0}
\definecolor{Pending}{rgb}{1.0,1.0,1.0}
\definecolor{Not Verified}{rgb}{1.0,0.0,0.0}
\definecolor{Verified}{rgb}{0.0,1.0,0.0}
\definecolor{Done}{rgb}{0.0,1.0,0.0}

%erweiterte Formattierung der Kopf- und Fußzeilen
\usepackage{fancyhdr}
\pagestyle{fancy}
\fancyhead[L]{\nouppercase{\leftmark}}
\fancyhead[R]{\thepage}
\fancyfoot{}
\renewcommand{\headrulewidth}{0.5pt}

% Fußzeile auf jeder Seite - auch Kapitel und Inhaltsverzeichnis
\fancypagestyle{plain}{
        \fancyhead[L]{\nouppercase{\leftmark}}
		\fancyhead[R]{\thepage}
		\renewcommand{\headrulewidth}{0.5pt}
}

\nonfrenchspacing % etwas mehr platz zwischen Sätzen


% Zur Bearbeitung von Tabellen - und Abbildungsüberschriften
\usepackage[format=plain,							% plain, hang
						labelsep=endash,					% none, colon, period, space, quad, newline
						justification=justified, 	% justified, centering, centerlast, 
																			% centerfirst, raggedright, raggedleft
						singlelinecheck=true,			% true, false
						width=0.75\textwidth,
						font=footnotesize,
						labelfont=bf,
						textfont=rm,
						labelfont=bf]{caption}


% Eigene Farben definieren																			
\usepackage{color}
\definecolor{darkblue}{rgb}{0,0,0.35}
\definecolor{webbrown}{rgb}{.6,0,0}

% Querverweise																			
% Falls die Ausgabe als PDF erfolgen soll, das Paket hyperref folgendermaßen
% einfügen
\usepackage[pdftex,
						plainpages=false,
						pdfpagelabels=true,
						hypertexnames=false
						]{hyperref}   

	\hypersetup{%
	%--- Farbige Links -----------------------------------------------------------
		colorlinks				= true,
		linkcolor					=	webbrown,
		urlcolor					=	webbrown,
	%--- Bookmarks ----------------------------------------------%
		bookmarksopen		= false,						% alle Gliederungsebenen offen?
		bookmarksnumbered	= true,							% Lesezeichen mit Abschnittsnummern
		linktocpage			= true,							% nur Seitenzahlen im TOC sind Links
	%--- Dokument Informationen ----------------------------------
		pdftitle 				= {Implementation of the Attitude Determination System of the Scintillation Event Triggering Hodoscope (SETH)},
		pdfsubject				= {Bachelor's Thesis},
		pdfauthor				=	{Nicolas Rohrbeck},
		pdfkeywords				=	{},
		pdfcreator				=	{Ghostscript ps2pdf},
		pdfproducer				= {LaTeX with hyperref},
	%--- Dokument Anzeige --------------------------------------------------------
		pdfpagemode				=	UseOutlines,			% Show Bookmarks in left frame  
		%pdfpagemode			ü=	None,							% No Bookmarks in left frame 
		pdfstartview			= Fit,							% fit whole page
		pdfstartpage			=	1,
		pdfpagelayout			=	OneColumn
		%pdfpagelayout	   = TwoColumnRight
}

% Damit auch subsection im table of content angezeigt werden
\setcounter{tocdepth}{2}

\newcommand{\un}[2]{#1\,\mathrm{#2}} 
\newcommand{\dd}{\mathrm{d}}


\usepackage[backend=biber, style=phys, urldate=comp, dateabbrev=false]{biblatex}
\addbibresource{bachelor.bib}

\usepackage{} % to put a box around text

\usepackage{geometry}
\geometry{
bottom = 3cm
}

\usepackage{mathtools}

\renewcommand{\deg}{$^\circ$}