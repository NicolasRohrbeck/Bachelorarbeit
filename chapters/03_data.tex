\chapter{Data \label{ch:data}}
In this chapter the necessary data for the calibration of the sensors is presented. Section \ref{sec:da:vector_fields} explains what information about the fields is used to calibration of the sensors.

\section{Vector fields \label{sec:da:vector_fields}}
The \ac{DOC} \ac{NOAA} has publicly available geomagnetic field calculators using the \acf{WMM}, \acf{IGRF}, \acf{EMM} or \acf{WMMHR}. Because no uncertainty is provided for the \ac{IGRF} values and the \ac{EMM} is only valid for the years 2000 to 2019 both of these models are not considered. The \ac{WMM} and \ac{WMMHR} values for Kiel (54\deg\,20'\,18''\,N 10\deg\,7'34''\,E) on 1\:April\:2025 at 0\,km above \ac{MSL} are presented in table \ref{tab:da:mag_models_comp}.

\begin{table}[h]
    \centering
    \begin{tabular}{r|ccc}
        Model & \ac{WMM} & \ac{WMMHR} & \ac{WMMHR} at 40\,km \ac{MSL}\\\hline
        Declination & 4\deg\,9'\,35''\,E & 4\deg\,12'\,27''\,E & 4\deg\,5'\,27''\,E \\
        North Comp. (mGs) & 177.8(1.4) & 178.4(1.4) & 175.6(1.4) \\ 
        East Comp. (mGs) & 12.9(0.9) & 13.1(0.9) & 12.6(0.9) \\
        Vertical Comp. (mGs) & 470.0(1.4) & 470.5(1.3) & 461.7(1.3) \\
        Total Field (mGs) & 502.6(1.4) & 503.4(1.3) & 494.2(1.3) \\
    \end{tabular}
    \caption[Comparison of the \acs{WMM} \cite{WMM} and \acs{WMMHR} \cite{WMMHR} in Kiel.]{Comparison of the \acs{WMM} \cite{WMM} and \acs{WMMHR} \cite{WMMHR}. Given in brackets are the uncertainties. All values are rounded to the error.}
    \label{tab:da:mag_models_comp}
\end{table}

Because of its lower uncertainties in all components of the magnetic field the \ac{WMMHR} is chosen for the calibration of the magnetometer. In light of the \ac{SETH} experiment the same table is given for Kiruna in the appendix \ref{sec:app:mag_field}.\\
As the difference between 0\,km and 40\,km above \ac{MSL} is non-negligible at 10\,mGs, the different values will be used when applicable. When fitting the ground based measurement, the value for 0\,km altitude ($(503.4\,\mathrm{mGs})^2=0.25341156\mathrm{(Gs)}^2$) is used and when fitting the flight, the value for 40\,km altitude ($(494.2\,\mathrm{mGs})^2=0.24423364\mathrm{(Gs)}^2$) is used.

The Accelerometer returns values in units of $g$, thus the magnitude of the field used for calibration is simply 1.

\section{Calibration Measurements \label{sec:da:calibration_meas}}
To find the coefficients $a$, $b$, $c$, $x_0$, $y_0$ and $z_0$ a set of three measurements are taken. The first measurement is taken in the \ac{IEAP}'s courtyard/car park on 11 April 2025. For this measurement, the sensors are rotated so that all faces of the casing lay on the ground once. After this a "random tumble" is performed, where the sensor is rotated randomly through all orientations at a height of about 1.7\,m above the ground.\\
Measurements two and three are also random tumbles at a height of about 1.7\,m. Measurement two is performed HITHER and measurement three TITHER.

\section{ID lines \label{sec:da:id_lines}}
The lines consist of IDs.

\section{Misalignment of Sensor Axes \label{sec:da:misalignment}}
If the half-axes of the ellipsoids are aligned with the x,y and z axes of the coordinate system, the misalignment is zero. If a half-axis is deflected from the coordinate axis, the sensors are misaligned.