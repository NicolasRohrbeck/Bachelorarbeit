\chapter{Conclusion and Outlook\label{ch:conclusion_outlook}}
We have shown that the accelerometer and magnetometer need to be calibrated to extract meaningful data. Multiple methods to determine a set of six coefficients per sensor were compared and the optimal methods determined. The optimal set of coefficients from these methods was used to calibrate the sensors post-flight. The calibrated dataset of magnetometer and accelerometer vectors was then used to calculate the gondolas pitch and roll angles as well as its heading. When plotting these it could be seen that the gondola is rotating quickly after take of and slowing down over time until the balloon bursts, when the gondola starts to wobble between North and ???.

It is the scientific objective of \ac{SETH} to accurately measure the angular dependency of the Regener-Pfotzer-Maximum on the polar and azimuth angle. As was shown in Chapter \ref{ch:flight_data}, the gondola is in constant motion during the flight turning and pitching uncontrollably. The \ac{ADS} that was presented in this thesis is therefore an absolute necessity for \ac{SETH} to reliably do its job. If the \ac{ADS} malfunctions during the flight no and no heading information is gathered at all, that is half of the scientific objective unfulfilled. The correct fabrication, integration and calibration of the sensors is thus directly linked to the overall success of the mission.
 

Because the magnetic environment caused by screws, housings or other hard iron sources in \ac{SETH} is fundamentally different from the junior flight or the laboratory, the calibration will have to be repeated for the finally assembled experiment.