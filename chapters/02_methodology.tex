\chapter{Methodology \label{ch:methodology}}
This chapter presents the hard- and software used in the \ac{ADS} as it was prepared for the junior flight of \ac{SETH}. The methodology covers a mixture of work done by senior scientists of the \ac{IEAP} and our own contributions. Most notable eternal contribution comes from Dr. Stephan Böttcher who did the \ac{PCB} design, wrote the \ac{FPGA} code and soldered the example which was used. 

\section{Data Acquisition \label{sec:meth:data_acquisition}}
During the flight multiple instruments are used to acquire data: an accelerometer and magnetometer for the attitude determination and three 10$\times$20\,mm$^2$ silicon \acfp{SSD}. The \acp{SSD} are used as a scientific payload to determine particle count rates and observe \acp{GCR}.

The program \verb|rpirena.py| is used to stream the data coming from the \ac{FPGA} to the SD\:Card or the \verb|Asterix| server. This program was written by Dr. Stephan Böttcher and is running on a Raspberry Pi Zero. As of March 14$^\mathrm{th}$ 2025 the program code can be found under \verb|asterix/home/subversion/stephan/solo/eda/cospi/host|. The program writes different event lines into a file, depending on the header of the packet taken from a register. The important lines for this thesis are \verb|ID| lines. These lines contain the information from the accelerometer and magnetometer, each entry separated with a white space. The first entry is the line identifier "ID", followed by the packet header 4805$\equiv$0x12C5. After this, the number of words in the line minus one (for the packet header) is printed. This number should be 132 by design. After this, the information words from accelerometer and magnetometer begin. An example of the line is given below.
\begin{lstlisting}
ID 4805 132 
\end{lstlisting} 

\section{Data Interpretation \label{sec:meth:data_interpretation}}
To interpret the data, instances of the \textit{ads} class from the \verb|python| script \verb|att_eval.py| are used. Each instance is an \textit{ads} object of a specific \verb|.EI| file. The whole file is read with python's built-in \textit{.lineread()} method. The parser then reads each entry until a housekeeping ("H") line is read. The timestamp (ts1) from this line is extracted into a variable. The parser iterates over the next lines until it reaches the next Housekeeping line whose timestamp (ts2) is also saved. Meanwhile the vectors of the "ID" lines are saved into $20\times6$ numpy arrays. The extracted vectors from the "ID" lines are then stacked into an array with $n\cdot20$ lines and 6 columns where n is an integer equal to the number of "ID" lines between the two "H" lines. The timestamps are calculated using \textit{numpy's .arange()} method with the start time (ts1) -1 [second] as the \textit{start} parameter and the end time (ts2) as the \textit{end} parameter\footnote{The arange method works until the value of the end parameter -1}. The step size is 0.1 which is equivalent to 100\,ms. The start time has to start at ts1$-$1 because the time that the vectors are measured and digitalized is not the same as the time when they are requested by the \ac{FPGA}.

The methods \textit{.plot\_vectors()}, \textit{.plot\_sphere()}, \textit{.plot\_angles()} and \textit{.plot\_heading()} are used to display the specified data. The data plotted is: The components of the accelerometer and magnetometer, converted to mGs and units of $g$, against time, the components in three dimensional space, the calculated pitch and roll angles against time and the calculated heading against time.

\section{Sensor Calibration Technique \label{sec:meth:calibration_technique}}
The technique applied to calibrate the magnetometer and accelerometer is presented in \cite{non-orthonogality}. The calibration relies only on the magnitude of the vector field to be measured. This is due to the insight that the flawless measurements of the sensors plotted in three dimensional space all fall onto the surface of a sphere. The radius of the sphere is simply the magnitude of the vector field to be measured. For the earth's magnetic field, $B_H$, this is:
\begin{align}
    \vec{B}_H&=\begin{pmatrix} B_x \\ B_y \\ B_z \end{pmatrix} \label{eq:earth_field} \\
    \iff B_H^2& = B_x^2+B_y^2+B_z^2 
    \label{eq:regular_sphere}
\end{align}

If we assume the measurement of each component of the field to be affected by a scale factor and a linear offset, we get the following three equations. The field to be measured is given without a subscript (as in \eqref{eq:earth_field}) and we use the superscript $B$ to denote that these are the offsets in the measurement of the magnetometer.
\begin{align}
    B_{x,\ meas} &= \frac{B_x}{\sqrt{a^B}}+x_0^B \\
    B_{y,\ meas} &= \frac{B_y}{\sqrt{b^B}}+y_0^B \\
    B_{z,\ meas} &= \frac{B_z}{\sqrt{c^B}}+z_0^B
\end{align}

Solving for $B_{x,y,z}$ in each of the formulae above gives us:
\begin{align}
    B_x&=\sqrt{a^B}(B_{x,\ meas}-x_0^B) \label{eq:bx} \\
    B_y&=\sqrt{b^B}(B_{y,\ meas}-y_0^B) \label{eq:by} \\
    B_z&=\sqrt{c^B}(B_{z,\ meas}-z_0^B) \label{eq:bz}
\end{align}

Plugging eqs. \eqref{eq:bx} to \eqref{eq:bz} into eq. \eqref{eq:regular_sphere} gives us the formula for an ellipsoid shifted off the origin by $x_0^B$, $y_0^B$, and $z_0^B$. The half-axes of the ellipsoid are given by the scale factors.
\begin{align}
    B_{H}^2&=a^B(B_{x,\ meas}-x_0^B)^2 + b^B(B_{y,\ meas}-y_0^B)^2 + c^B(B_{z,\ meas}-z_0^B)^2 
    \label{eq:mag_fit_function}
\end{align}

 Equation \eqref{eq:mag_fit_function} is used to determine the coefficients $a^B$, $b^B$, $c^B$, $x_0^B$, $y_0^B$ and $z_0^B$ using \verb|gnuplot|'s fit function. The measured vectors are given as input for the right hand side of the equation while the square of the magnitude of the earth's magnetic field in Kiel, given as 0.2488 Gs, is used for the left hand side.\\
It is now also apparent, why the scale factors in eqs. \eqref{eq:bx} - \eqref{eq:bz} are chosen as one over the square root: the fit converges faster and more accurately when only fitting a linear relationship.

While eq. \eqref{eq:mag_fit_function} has been derived for the earth's magnetic field, the same concept is used in the calibration of the accelerometer. The function used to determine the coefficients is eq. \eqref{eq:acc_fit_function}. Because the data is in units of $g$ the magnitude of the field is 1.
\begin{equation}
    1\overset{!}{=}a^g(g_{x,\ meas}-x_0^g)^2 + b^g(g_{y,\ meas}-y_0^g)^2 + c^g(g_{z,\ meas}-z_0^g)^2
    \label{eq:acc_fit_function}
\end{equation}

After the coefficients $a$, $b$, $c$, $x_0$. $y_0$ and $z_0$ are determined for accelerometer and magnetometer, the measured data set is modified as given in eqs. \eqref{eq:bx} - \eqref{eq:bz} to get the true components without measurement error.


% OPTIONAL: To interpret these mathematical coefficients in a physical sense, the matrices in eqs. \eqref{eq:bg:g_with_errors} and \eqref{eq:bg:b_with_errors} from sec. \ref{sec:bg:measurement_errors} have to be compared to the numerically found coefficients. This complex relationship is shown below in eqs. \ref{eq:coeff_relation_a} and has been derived in appendix \ref{sec:app:deriv_of_coeff}.

\section{Determination of Heading \label{sec:meth:determination_heading}}
The calibrated dataset is used to determine the gondolas heading. For this the magnetic field vector in the body frame ($bf$) of reference is translated to a world frame ($wf$) using the accelerometer measurement. After this translation, the components all lie in the correct planes. The last open variable is the orientation of the x axis in the world frame. which is determined using the magnetometer. The acceleration vector in the world frame is given in \eqref{eq:meth:g_vector}. Note that the magnitude of $\vec{g}$ is 1 because measurements are in units of $g$.
\begin{equation}
    \vec{g}^{\ wf}=\begin{pmatrix} 0 \\ 0 \\ 1 \end{pmatrix} \label{eq:meth:g_vector}
\end{equation}

If the gondola is deflected by an angle, the measurement of the gravity vector will have non zero $x$ or $y$ components. Pitch or roll of the gondola are determined by calculating the arctangent of the no zero component divided by the $z$ component. Using $x$, the result is the pitch angle (i.e. rotation about the y-axis) and using $y$ the result is the roll angle (i.e. rotation about the x-axis).\\
If the deflection is not bound to only the x- or y-axis, spherical coordinates are used to describe the gravity vectors position. We define the angle between the projected vector in the xy-plane and the x-axis as $\phi\in[-\pi,\pi]$. The angle between the z-axis and the vector is defined as $\theta\in[0,\pi]$. Positive angles are counted counter-clockwise, consistent with the right hand rule.

To rotate the body frame to the world frame, a rotation of $\phi$ is performed about the z-axis to align the body frame's x-axis with the deflection of the gravity vector. A second rotation of angle $\theta$ about the y-axis aligns the z-axis with the gravity vector. The world frame's xy-plane is now perpendicular to the earth's surface or equivalently: The positive z-axis is now anti-parallel to the normal vector of the earth's surface (i.e. $-\vec{g}$). The direction of the world frame's x-axis is rotated randomly about the z-axis and points in the (random) direction of the gondolas deflection. To determine what direction this is, the measured magnetometer vector is rotated to the world frame as shown below.
\begin{equation}
    \vec{B}_{meas}^{\ wf}=R_y(\theta)R_z(\phi)\vec{B}_{meas}^{\ bf}
\end{equation}
With:
\begin{align}
    R_z(\phi)&=\begin{pmatrix}
                \cos\phi & -\sin\phi & 0 \\
                \sin\phi & \cos\phi & 0 \\
                0 & 0 & 1
                \end{pmatrix} \\
    R_y(\theta)&=\begin{pmatrix}
                \cos\theta & 0 & \sin\theta \\
                0 & 1 & 0 \\
                -\sin\theta & 0 & \cos\theta
                \end{pmatrix}
\end{align}

Seeing as Magnetic North is the direction in which the xy-projection of the magnetic field is pointing, the xy-projection of $\vec{B}_{meas}^{\ wf}$ needs to be evaluated. The last angle $\psi\in[0,2\pi]$ is the angle between the x-axis of the world frame and the xy-projection of the magnetic field vector minus the declination and equivalent to the gondolas heading. The declination $\delta$ of the magnetic field describes the location dependant angle between True and Magnetic North.

By computing pitch, roll and heading as described above, objectives 1 and 2 of this thesis are fulfilled.

\section{Reflection on Methodology \label{sec:meth:reflection_methodology}}