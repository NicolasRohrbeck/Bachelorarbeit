\addchap{Introduction \label{ch:introduction}}
The \acs{REXUS}/\acs{BEXUS} programme is realised under a bilateral Agency Agreement between the \acf{DLR} and the \acf{SNSA}. The Swedish share of the payload has been made available to students from other European countries through a collaboration with the \acf{ESA}. EuroLaunch, a cooperation between the \acf{SSC} and the \acf{MORABA} of \acs{DLR}, is responsible for the campaign management and operations of the launch vehicles. Experts from \acs{DLR}, \acs{SSC}, \acs{ZARM} and \acs{ESA} provide technical support to the student teams throughout the project. \acs{REXUS} and \acs{BEXUS} are launched from \acs{SSC}, \acs{ESRANGE} Space Center in northern Sweden.

The first balloon of the \acs{REXUS}/\acs{BEXUS} programme, \textit{BEXUS 1}, launched on November 25$^{\mathrm{th}}$ 2002 at 15:53 UTC. Every year after this, barred for 2003, 2020 and 2022, saw balloons being launched from the designated launch site \ac{ESRANGE} near the Swedish city of Kiruna \parencite{IAC-08.E.1.1.4}\parencite{bexus-campaign-history}.
The \ac{REXUS}/\ac{BEXUS} programme offers university student teams the opportunity to build their own experiment and have it fly on a stratospheric balloon. The programme spans around one year and is constructed as a scaled-down version of a real space mission. \ac{SETH} takes part in the 16th \ac{BEXUS} cycle and is expected to fly between the 3$^\mathrm{rd}$ of October 2025 and 13$^\mathrm{th}$ of October 2025 from \ac{ESRANGE}.

This thesis will cover a sub-system of \ac{SETH} with the purpose of measuring two vectors: the magnetic field of the earth and the gravitational acceleration. At first we will give necessary background information about the \ac{SETH} experiment as a part of the \ac{REXUS}/\ac{BEXUS} programme and explain the working principles of the used magnetometer and accelerometer and the possible sources of systematic measurement error.\\
Chapter \ref{ch:methodology} presents the foundations of what is done during data acquisition (sec. \ref{sec:meth:data_acquisition}), the program used to calculate and plot the results (sec. \ref{sec:meth:data_interpretation}), an extensive overview of how the sensors are calibrated (sec. \ref{sec:meth:calibration_technique}), how pitch, roll and heading are calculated (sec. \ref{sec:meth:determination_heading}) and some reflection on the previous steps after the calibration is done (sec. \ref{sec:meth:reflection_methodology}).\\
Chapter \ref{ch:data} presents the data which was used for calibration and how the calibration measurement was done.\\
Chapter \ref{ch:calibration_results} presents the results of the calibration measurement and how the optimal coefficients were acquired.\\
In chapter \ref{ch:flight_data} the optimal coefficients are used to calibrate the data set acquired during flight and an analysis is done on pitch, roll and heading around launch, during ascent, around balloon burst and during descent.\\
Chapter \ref{ch:conclusion_outlook} will give a short summary of the work and will touch on its relevancy to the \ac{SETH} experiment.